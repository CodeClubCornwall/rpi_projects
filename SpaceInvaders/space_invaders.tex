%%%%%%%%%%%%%%%%%%%%%%%%%%%%%%%%%%%%%%%%%
% Programming/Coding Assignment
% LaTeX Template
%
% This template has been downloaded from:
% http://www.latextemplates.com
%
% Original author:
% Ted Pavlic (http://www.tedpavlic.com)
%
% Note:
% The \lipsum[#] commands throughout this template generate dummy text
% to fill the template out. These commands should all be removed when 
% writing assignment content.
%
% This template uses a Perl script as an example snippet of code, most other
% languages are also usable. Configure them in the "CODE INCLUSION 
% CONFIGURATION" section.
%
%%%%%%%%%%%%%%%%%%%%%%%%%%%%%%%%%%%%%%%%%

%----------------------------------------------------------------------------------------
%	PACKAGES AND OTHER DOCUMENT CONFIGURATIONS
%----------------------------------------------------------------------------------------

\documentclass{article}

\usepackage{fancyhdr} % Required for custom headers
\usepackage{lastpage} % Required to determine the last page for the footer
\usepackage{extramarks} % Required for headers and footers
\usepackage[usenames,dvipsnames]{color} % Required for custom colors
\usepackage{graphicx} % Required to insert images
\usepackage{listings} % Required for insertion of code
\usepackage{courier} % Required for the courier font
\usepackage{framed}
\usepackage{wasysym}
\usepackage{tcolorbox}
%\usepackage{lipsum} % Used for inserting dummy 'Lorem ipsum' text into the template

% Margins
\topmargin=-0.45in
\evensidemargin=0in
\oddsidemargin=0in
\textwidth=6.5in
\textheight=9.0in
\headsep=0.25in

\linespread{1.1} % Line spacing

% Set up the header and footer
\pagestyle{fancy}
%\lhead{\hmwkAuthorName} % Top left header
%\chead{\hmwkClass\ (\hmwkClassInstructor\ \hmwkClassTime): \hmwkTitle} % Top center head
\chead{\hmwkClass: \hmwkTitle} % Top center head
%\rhead{\firstxmark} % Top right header
\rhead{} % Top right header
\lfoot{\lastxmark} % Bottom left footer
\cfoot{} % Bottom center footer
\rfoot{Page\ \thepage\ of\ \protect\pageref{LastPage}} % Bottom right footer
\renewcommand\headrulewidth{0.4pt} % Size of the header rule
\renewcommand\footrulewidth{0.4pt} % Size of the footer rule

\setlength\parindent{0pt} % Removes all indentation from paragraphs

%----------------------------------------------------------------------------------------
%	CODE INCLUSION CONFIGURATION
%----------------------------------------------------------------------------------------

\definecolor{MyDarkGreen}{rgb}{0.0,0.4,0.0} % This is the color used for comments
\lstloadlanguages{Python} % Load Perl syntax for listings, for a list of other languages supported see: ftp://ftp.tex.ac.uk/tex-archive/macros/latex/contrib/listings/listings.pdf
\lstset{language=Python, % Use Perl in this example
        frame=single, % Single frame around code
        basicstyle=\small\ttfamily, % Use small true type font
        keywordstyle=[1]\color{Blue}\bf, % Perl functions bold and blue
        keywordstyle=[2]\color{Purple}, % Perl function arguments purple
        keywordstyle=[3]\color{Blue}\underbar, % Custom functions underlined and blue
        identifierstyle=, % Nothing special about identifiers                                         
        commentstyle=\usefont{T1}{pcr}{m}{sl}\color{MyDarkGreen}\small, % Comments small dark green courier font
        stringstyle=\color{Purple}, % Strings are purple
        showstringspaces=false, % Don't put marks in string spaces
        tabsize=5, % 5 spaces per tab
        %
        % Put standard Perl functions not included in the default language here
        morekeywords={rand},
        %
        % Put Perl function parameters here
        morekeywords=[2]{on, off, interp},
        %
        % Put user defined functions here
        morekeywords=[3]{test},
       	%
        morecomment=[l][\color{Blue}]{...}, % Line continuation (...) like blue comment
        numbers=left, % Line numbers on left
        firstnumber=1, % Line numbers start with line 1
        numberstyle=\tiny\color{Blue}, % Line numbers are blue and small
        stepnumber=5 % Line numbers go in steps of 5
}

% Create new commands to include python and bash scripts. The first parameter
% is the filename of the script without .py or .sh and the second parameter
% is the caption.
\newcommand{\pythonscript}[2]{
\begin{itemize}
\item[]\lstinputlisting[caption=#2,label=#1]{#1.py}
\end{itemize}
}
\newcommand{\bashscript}[2]{
\begin{itemize}
\item[]\lstinputlisting[caption=#2,label=#1]{#1.sh}
\end{itemize}
}

%----------------------------------------------------------------------------------------
%	DOCUMENT STRUCTURE COMMANDS
%	Skip this unless you know what you're doing
%----------------------------------------------------------------------------------------

% Header and footer for when a page split occurs within a problem environment
\newcommand{\enterProblemHeader}[1]{
\nobreak\extramarks{#1}{#1 continued on next page\ldots}\nobreak
\nobreak\extramarks{#1 (continued)}{#1 continued on next page\ldots}\nobreak
}

% Header and footer for when a page split occurs between problem environments
\newcommand{\exitProblemHeader}[1]{
\nobreak\extramarks{#1 (continued)}{#1 continued on next page\ldots}\nobreak
\nobreak\extramarks{#1}{}\nobreak
}

\setcounter{secnumdepth}{0} % Removes default section numbers
\newcounter{homeworkProblemCounter} % Creates a counter to keep track of the number of problems

\newcommand{\challengeStepName}{}
\newenvironment{challengeStep}[1][Step \arabic{homeworkProblemCounter}]{ % Makes a new environment called homeworkProblem which takes 1 argument (custom name) but the default is "Problem #"
\stepcounter{homeworkProblemCounter} % Increase counter for number of problems
\renewcommand{\challengeStepName}{#1} % Assign \homeworkProblemName the name of the problem
\section{\challengeStepName} % Make a section in the document with the custom problem count
\enterProblemHeader{\challengeStepName} % Header and footer within the environment
}{
\exitProblemHeader{\challengeStepName} % Header and footer after the environment
}

\newcommand{\problemAnswer}[1]{ % Defines the problem answer command with the content as the only argument
\noindent\framebox[\columnwidth][c]{\begin{minipage}{0.98\columnwidth}#1\end{minipage}} % Makes the box around the problem answer and puts the content inside
}

\newcommand{\homeworkSectionName}{}
\newenvironment{homeworkSection}[1]{ % New environment for sections within homework problems, takes 1 argument - the name of the section
\renewcommand{\homeworkSectionName}{#1} % Assign \homeworkSectionName to the name of the section from the environment argument
\subsection{\homeworkSectionName} % Make a subsection with the custom name of the subsection
\enterProblemHeader{\challengeStepName\ [\homeworkSectionName]} % Header and footer within the environment
}{
\enterProblemHeader{\challengeStepName} % Header and footer after the environment
}

%----------------------------------------------------------------------------------------
%	NAME AND CLASS SECTION
%----------------------------------------------------------------------------------------

\newcommand{\hmwkTitle}{Space Invaders} % Assignment title
\newcommand{\hmwkDueDate}{Monday,\ January\ 1,\ 2012} % Due date
\newcommand{\hmwkClass}{Code Club Challenge} % Course/class
\newcommand{\hmwkClassTime}{10:30am} % Class/lecture time
\newcommand{\hmwkClassInstructor}{Jones} % Teacher/lecturer
\newcommand{\hmwkAuthorName}{John Smith} % Your name

%\newenvironment{hint}{
%  \begin{framed}
%  \textbf{Hint}
%
%}{\end{framed}} % TODO put this in a box

\definecolor{mycol}{rgb}{0.122, 0.435, 0.698}
\newenvironment{hint}{
  \begin{tcolorbox}[colback=green!5,colframe=green!40!black,title=Note]}
{\end{tcolorbox}}

%----------------------------------------------------------------------------------------
%	TITLE PAGE
%----------------------------------------------------------------------------------------

\title{
\vspace{2in}
\textmd{\textbf{\hmwkClass:\ \hmwkTitle}}\\
%\normalsize\vspace{0.1in}\small{Due\ on\ \hmwkDueDate}\\
%\vspace{0.1in}\large{\textit{\hmwkClassInstructor\ \hmwkClassTime}}
\vspace{3in}
}

%\author{\textbf{\hmwkAuthorName}}
\date{} % Insert date here if you want it to appear below your name

%----------------------------------------------------------------------------------------

\begin{document}

\maketitle

%----------------------------------------------------------------------------------------
%	TABLE OF CONTENTS
%----------------------------------------------------------------------------------------

%\setcounter{tocdepth}{1} % Uncomment this line if you don't want subsections listed in the ToC

\newpage
%\tableofcontents
%\newpage

%----------------------------------------------------------------------------------------
%	PROBLEM 1
%----------------------------------------------------------------------------------------

\section{Setting up}

One of the very first computer games was called "Space Invaders"\footnote{It
was apparently originally created in 1978, which makes it almost as old as me!}
and we are going to try to recreate it in Python.

To get started, open LXTerminal and use the following command to download the
start of the project:

\bashscript{setup}{Get the the starting point}

This should create a new directory called \texttt{space\_invaders} containing
a Python program and some icons. Open the program with:

\bashscript{openscript}{Opening the starting point}

\begin{hint}
We need to use idle (for Python version 2) instead of idle3 (for Python
version 3) because the PyGame module only works for Python version 2 on the
Raspberry Pi at the moment.

You could also launch Idle from the desktop icon and then browse to the
correct directory to open the file.
\end{hint}

\section{Controlling the ship}

The first thing to do is to add the ship that you (the player) will control.
This will move left and right along the bottom of the screen while the aliens,
which you have to shoot, will go at the top. Look at the script below and see
what you need to change in the program you have written in the first step.

Can you identify the parts that:

\begin{itemize}
  \item Finds the picture file to use for the ship?
  \item Tells the computer where to put the ship?
  \item Finds out if you are pressing the left or right buttons?
\end{itemize}

%\pythonscript{step2}{Controlling the ship}

\section{Make the movement easier}

You have probably noticed that the movement of the ship is not very easy. You
really want to be able to hold the key pressed rather than having to press it
lots of times. The reason it is acting the way it is is because when it
detects a keyboard \emph{event} with \texttt{event.type == pygame.KEYDOWN} it sees
the single action when you press the key and moves the ship once. What we
really want is for it to tell when the key is pressed and then keep moving
the ship until it is released, with \texttt{event.type == pygame.KEYUP}.

%\pythonscript{step3}{Easier control of the ship}

\section{Add a bit of \emph{class}}

Now we have managed to get the ship moving more easily but do you see what
happens when you go to the edge of the window? We will solve this later.
First, we are going to introduce a new
programming idea; \emph{Classes.}

So far, we have just a single thing in our game -- a ship that you can control
with the arrow keys. Later we will also have a number of aliens as well as
bullets, and all of these have very similar information about them and actions.
For example, they all have a position (x and y) and they all need to move.
With a Class we can write code for them all once and only once.

\begin{hint}
Note, some of the lines marked ``Changed'' have very small changes, even just a
single character, while others replace several lines with just one.

Can you see how I stopped the ship going off the side of the window?

Can you work out how to make it move faster or slower?
\end{hint}

%\pythonscript{step4}{Adding classes}

\section{Explanation}

In the previous section we added a \emph{Class} called \texttt{GamePiece}.
By itself, this does not do anything to it provides a pattern from which
we created an \emph{Object} for the ship. Later, we will add another object
for an alien and our code will contain:

%\pythonscript{explanation1}{Object creation example}

Each of these is separate from the other and has its own x and y coordinates.
They also both know how to move and draw themselves in the correct position
in the window if we call the functions:

%\pythonscript{explanation2}{Function call example}

and we can even set their speeds with:

%\pythonscript{explanation3}{Setting object variables}

\begin{hint}
This may be the answer to one of the questions in the last section!
\end{hint}

As well as the \texttt{move} and \texttt{draw} functions you will see another
function called \texttt{\_\_init\_\_} (that is a double `\_' before and after
the word `init'. This is a special function that is always run once when the
object is created with \texttt{ship = GamePiece(320, 410, ship\_image)}.
This makes sure that all the variables have the correct values at the start.

\section{Add an alien}

The real Space Invaders game has lots of aliens in rows but for the moment we
will start with just one. As I said in the last section, a lot of the work is
already done for us as we can use the \texttt{GamePiece} class.

\begin{hint}
I am not going to write out the whole code from now, only the parts that need
to change.
\end{hint}

First, make some changes to the \texttt{GamePiece} class to let the alien
``bounce'' when it reaches the side of the screen.

%\pythonscript{step5changes1}{Changes to the class}

Next, this is all that is needed to create the alien.

%\pythonscript{step5changes2}{Create the alien object}

%\pythonscript{step5changes3}{Use the alien object}

\section{Shoot the alien}

The next thing to do is to let you try to shoot the alien. For this, we need
to load an image of a bullet with this line (try to find the correct place to
put it):

%\pythonscript{step6bulletimage}{Load the bullet image}

Now the bullet needs to move up the screen, rather than left or right, and
then disappear when it reaches the top. For this we need to add some new
variables to the \texttt{\_\_init\_\_} function:

%\pythonscript{step6init}{New variables}

and add a new part to the \texttt{move} function:

%\pythonscript{step6move}{New move function}

At the start of the game, the bullet doesn't exist and we can indicate this
with:

%\pythonscript{step6initialisebullet}{Initialise bullet}

And finally, the bullet should appear when we press the `SPACE' bar.

%\pythonscript{step6fire}{Fire!}

\section{Hit the alien}

You may have noticed that the bullet currently just goes straight through the
alien. This isn't very good! In the \texttt{GamePiece} class we need a new
function to detect if it has been hit.

%\pythonscript{step7hit}{Detect a hit}

At the moment there is only one alien so when it is hit we will put it back
at the top and make it move faster. Later, we can have several rows of aliens
and try to keep a score.

%\pythonscript{step7hit2}{Hit the alien}

\section{More aliens}

TODO

and lose if aliens reach the bottom.

\section{Keep score}

TODO

\section{Aliens' bullets}

TODO

\section{Next levels}

TODO

\end{document}
